\begin{homeworkProblem}[Maximum Subarray Sum (10 points)]

    Given an array $A$ of $n$ integers, the maximum subarray sum is the largest 
    sum of any contiguous subarray of $A$ (including the empty subarray). In 
    other words, the maximum subarray sum is:
    
    \[
        \max_{i \leq j} \sum_{k=i}^{j} A[k]
    \]

    For example, the maximum subarray sum of $[-2, 1, -3, 4, -1, 2, 1, -5, 4]$ 
    is 6, the sum of the contiguous subarray $[4, -1, 2, 1]$.

    Design an $O(n \log n)$-time divide-and-conquer algorithm that finds the 
    maximum subarray sum. Briefly explain why your algorithm is correct and 
    justify its running time.

    \solution \\
    Below is my algorithm description of a divide-and-conquer algorithm 
    \alg{MaxSubarraySum(A)} presented in pseudocode.

    \begin{algorithm}[h]
        \begin{algorithmic}[1]
            \Function{MaxSubarraySum}{$A$}
                \If{$A$ is empty}
                    \State{} \Return{0}
                \EndIf{}
                \If{$A$ has one element}
                    \State{} \Return{$\max(0, A[0])$}
                \EndIf{}
                \State{} $mid \gets \lfloor |A| / 2 \rfloor$
                \State{} $L \gets A[0 \ldots mid-1]$, $R \gets A[mid \ldots |A|-1]$
                \State{} $leftMax \gets$ \Call{MaxSubarraySum}{$L$}
                \State{} $rightMax \gets$ \Call{MaxSubarraySum}{$R$}
                \State{}
                \State{} $maxStart \gets -\infty$, $sum \gets 0$
                \For{$i$ from $|L|-1$ down to $0$}
                    \State{} $sum \gets sum + L[i]$
                    \State{} $maxStart \gets \max(maxStart, sum)$
                \EndFor{}
                \State{}
                \State{} $maxEnd \gets -\infty$, $sum \gets 0$
                \For{$i$ from $0$ to $|R|-1$}
                    \State{} $sum \gets sum + R[i]$
                    \State{} $maxEnd \gets \max(maxEnd, sum)$
                \EndFor{}
                \State{}
                \State{} \Return{$\max(leftMax, rightMax, maxStart + maxEnd)$}
            \EndFunction{}
        \end{algorithmic}
    \end{algorithm}

    The algorithm \alg{MaxSubarraySum(A)} produces the correct result on an 
    array of $n$ integers because each call selects the maximum over the 
    maximum subarray sums for the left and right halves of the $n$ integers and
    the maximum subarray sum of an array that crosses over the two halves. To 
    find the crossover subarray sum, the algorithm scans right to left for the 
    left half to find the maximum contribution from that half, and vice versa
    for the right half. The subproblems build up and select maximum sums from 
    subarrays that are entirely in each half or crossover.

    The following recurrence relation describes the runtime of the algorithm:

    \[
        T(n) = 2 T(\tfrac{n}{2}) + O(n)
    \]

    At each level of recursion finding the crossover subarray sum executes
    in $O(n)$ time, since the for loops iterate over the entire length of $A$
    once. The algorithm partitions $A$ into left and right halves on each
    recursion. Following from Master Theorem, the runtime is $O(n \log n)$ 
    since $\log_{b} a = d$ where $a = 2$, $b = 2$, and $d = 1$. 



\end{homeworkProblem}
