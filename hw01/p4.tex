\begin{homeworkProblem}[Counting Steps (Solo Question)]
    You can climb a ladder with $n$ rungs by climbing either 1 rung or 2 rungs 
    in each step. How many distinct ways are there to climb to the top?
    \\

    \part

    Give a simple recursive algorithm to compute the answer.
    \\

    \begin{homeworkAnswer}[\solution]

        Below is a simple recursive algorithm, \alg{Ways($n$)}, to compute how
        many distinct ways there are to climb to the top of a ladder with $n$
        rungs by climbing either 1 rung or 2 rungs in each step:
        
        \smallskip
        
        \begin{algorithmic}[1]
            \Function{Ways}{$n$}
                \If{$n = 0$}
                    \State{} \Return{$1$}
                \ElsIf{$n < 0$}
                    \State{} \Return{$0$}
                \Else{}
                    \State{} \Return{$\Call{Ways}{n - 1} + \Call{Ways}{n - 2}$}
                \EndIf{}
            \EndFunction{}
        \end{algorithmic}

    \end{homeworkAnswer}

    \pagebreak

    \part

    Prove correctness using induction.
    \\
    
    \begin{homeworkAnswer}[\solution]
        
        Below is a proof of correctness using induction on $n$ that the 
        algorithm correctly computes how many distinct ways there are to climb 
        a ladder with $n$ rungs by climbing either 1 rung or 2 rungs in each 
        step.
        
        \begin{proof}
            \textbf{Base cases:} 
                
            \begin{itemize}
                \item If $n = 0$, there is 1 way to climb the ladder, i.e., do 
                    nothing, and \textsc{Ways}$(0)$ returns 1.
                
                \item If $n < 0$, there are 0 ways to climb the ladder since it
                    has negative steps, and \textsc{Ways}$(0)$ returns 0.
            \end{itemize}

            
            \textbf{Inductive hypothesis:} Assume that for all $0 \leq m \leq 
            k$, \textsc{Ways}$(m)$ outputs the correct result.
            \\

            \textbf{Inductive step:} For $n = k + 1$, the algorithm returns 
            $\textsc{Ways}(k) + \textsc{Ways}(k- 1)$. By the inductive 
            hypothesis, $\textsc{Ways}(k)$ correctly counts ways to reach 
            rung $k$, from which it is possible to take 1 more step, and 
            $\textsc{Ways}(k - 1)$ correctly counts ways to reach rung 
            $k - 1$, from which we it is possible to take 2 more steps. 
            Their sum then counts all distinct ways there are to reach rung 
            $k + 1$, i.e., the top, by climbing either 1 rung or 2 rungs 
            in each step. 
        \end{proof}

    \end{homeworkAnswer}

\end{homeworkProblem}