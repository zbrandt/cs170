\begin{homeworkProblem}[Counting Steps (Solo Question)]
    You can climb a ladder with $n$ rungs by climbing either 1 rung or 2 rungs 
    in each step. How many distinct ways are there to climb to the top?
    
    \begin{itemize}
        \item[(a)] Give a simple recursive algorithm to compute the answer.
        \item[(b)] Prove correctness using induction.
    \end{itemize}

    \begin{homeworkAnswer}[\solution]
        \begin{itemize}
            \item[(a)] Below is a simple recursive algorithm to compute the 
                answer.
                
                {\color{blue}
                \begin{algorithmic}[1]
                    \Function{Steps}{$n$}
                        \If{$n = 0$}
                            \State{} \Return{$1$}
                        \ElsIf{$n < 0$}
                            \State{} \Return{$0$}
                        \Else{}
                            \State{} \Return{$\Call{Steps}{n - 1} + \Call{Steps}{n - 2}$}
                        \EndIf{}
                    \EndFunction{}
                \end{algorithmic}}

            \item[(b)] \begin{proof}
                I'll prove by induction on $n$ that the algorithm correctly 
                computes how many distinct ways there are to climb a ladder
                with $n$ rungs by climbing either 1 rung or 2 rungs in each
                step.

                \textbf{Base case:} If $n = 0$, there is 1 way to climb
                the ladder, i.e., do nothing, and \textsc{Steps}$(0)$ 
                returns 1.
                
                \textbf{Base case:} If $n < 0$, there are 0 ways to climb the
                ladder, and \textsc{Steps}$(0)$ returns 0.

                \textbf{Inductive hypothesis:} Assume that for all $0 \leq m \leq 
                k$, \textsc{Steps}$(m)$ outputs the correct result.

                \textbf{Inductive step:} For $n = k + 1$, the algorithm returns 
                $\textsc{Steps}(k) + \textsc{Steps}( k- 1)$. By the inductive 
                hypothesis, $\textsc{Steps}(k)$ correctly counts ways to reach 
                rung $k$, from which it is possible to take 1 more step, and 
                $\textsc{Steps}(k - 1)$ correctly counts ways to reach rung 
                $k - 1$, from which we it is possible to take 2 more steps. 
                Their sum then counts all distinct ways there are to reach rung 
                $k + 1$, i.e., the top, by climbing either 1 rung or 2 rungs 
                in each step. \end{proof}
        \end{itemize}
    
    \end{homeworkAnswer}

\end{homeworkProblem}