\begin{homeworkProblem}[Counting Steps (Solo Question)]
    You can climb a ladder with $n$ rungs by climbing either 1 rung or 2 rungs 
    in each step. How many distinct ways are there to climb to the top?
    
    \begin{itemize}
        \item[(a)] Give a simple recursive algorithm to compute the answer.

        \item[(b)] Prove correctness using induction.

    \end{itemize}

\end{homeworkProblem}

\begin{homeworkAnswer}[\solution]
    \begin{itemize}
        \item[(a)] Below is a simple recurisve algorithm to compute the answer.
            {\color{blue}
            \begin{algorithmic}[1]
                \Function{Steps}{$n$}
                    \If{$n = 0$}
                        \State{} \Return{$1$}
                    \ElsIf{$n < 0$}
                        \State{} \Return{$0$}
                    \Else{}
                        \State{} \Return{$\Call{Steps}{n - 1} + \Call{Steps}{n - 2}$}
                    \EndIf{}
                \EndFunction{}
            \end{algorithmic}}

        \item[(b)] \begin{proof}
            We prove by induction on $n$ that the algorithm correctly computes 
            the number of distinct ways to climb $n$ rungs.

            \textbf{Base cases:}
            \begin{itemize}
                \item When $n = 0$: There is exactly 1 way to climb 0 rungs (do 
                    nothing), and \textsc{Steps}$(0)$ returns 1. \checkmark
                \item When $n < 0$: There are 0 ways to climb a negative number 
                    of rungs, and \textsc{Steps}$(n)$ returns 0. \checkmark
            \end{itemize}

            \textbf{Inductive hypothesis:} Assume that for all $0 \leq m \leq 
            k$, \textsc{Steps}$(m)$ correctly computes the number of ways to 
            climb $m$ rungs.

            \textbf{Inductive step:} For $n = k + 1$, the algorithm returns 
            $\textsc{Steps}(k) + \textsc{Steps}(k-1)$. By the inductive 
            hypothesis, $\textsc{Steps}(k)$ correctly counts ways to reach rung 
            $k$ (from which we take 1 more step), and $\textsc{Steps}(k-1)$ 
            correctly counts ways to reach rung $k-1$ (from which we take 2 
            more steps). Their sum counts all distinct ways to reach rung 
            $k+1$. \end{proof}

    \end{itemize}

\end{homeworkAnswer}