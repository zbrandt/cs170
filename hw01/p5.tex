\begin{homeworkProblem}[Counting Steps Efficiently]
    \begin{itemize}
        \item[(a)] Analyze the running time of your algorithm from the previous 
            question. Is it bounded by a polynomial in $n$? Assume for 
            simplicity that two integers can be added in one timestep.

        \item[(b)] Give an efficient algorithm to compute the answer, and show 
            the running time is bounded by $O(n)$ (polynomial in $n$ is also 
            acceptable). 

        \item[(c)] (Extra credit) Give an even more efficient algorithm to 
            compute the answer, and analyze its running time, which should be 
            sub-polynomial, again assuming each integer arithmetic operation 
            takes $1$ step. {\it Hint: can you compute the answer by 
            multiplying $2\times 2$ matrices?}

            Is this running time a fair assessment of how the algorithm will 
            perform in practice?

    \end{itemize}

    \begin{homeworkAnswer}[\solution]
        \begin{itemize}
            \item[(a)] The recurrence equation for the algorithm is $T(n) = 
                T(n - 1) + T(n - 2) + O(n)$. No it is not bounded in polynomial
                time because the tree grows approximately exponentially in size
                with each increase in $n$.

            \item[(b)] Below is an efficient algorithm to compute the answer.
                {\color{blue}
                \begin{algorithmic}[1]
                    \Function{Steps}{$n$}
                        \State{$\text{rungs}[0] \gets 1$}
                        \State{$\text{rungs}[1] \gets 1$}
                        \For{$i \gets 2$ \textbf{to} $n$}
                            \State{$\text{rungs}[i] \gets \text{rungs}[i - 1] + \text{rungs}[i - 2]$}
                        \EndFor{}
                        \State{}\Return{$\text{rungs}[n]$}
                    \EndFunction{}
                \end{algorithmic}}

                
                Let $C(n)$ denote the number of iterations of the for-loop in 
                $\mathrm{Steps}(n)$. For $n \le 1$, the loop does not execute, 
                so $C(n) = 0$. For $n \ge 2$, the loop runs for $i = 2, 3, 
                \dots, n$, hence $C(n) = n - 1$. Each iteration performs one 
                addition and a 
                constant number of assignments, so costs $O(1)$ time. The 
                initialization of $\text{rungs}[0]$ and $\text{rungs}[1]$ also 
                costs $O(1)$. Therefore, the total running time is $O(1) + C(n) 
                \cdot O(1) = O(n)$. The space usage is $O(n)$ to store 
                $\text{rungs}[0..n]$.
            
            \item[(c)] Below is an even more efficient algorithm to compute the
                answer:

                {\color{blue}
                \begin{algorithmic}[1]
                    \Function{Steps}{$n$}
                        \State{$A \gets \left[\begin{array}{cc}1 & 1\\ 1 & 0\end{array}\right]$}
                        \State{$A \gets \Call{MatrixPower}{A,\; n + 2}$}
                        \State{}\Return{$A[1,1]$}
                    \EndFunction{}
                \end{algorithmic}}

                The above algorithm produces the correct answer on an arbitrary 
                input of $n \geq 1$. The function leverages matrix 
                multiplication of a $2 \times 2$ matrix, $A$, by computing 
                $A^{n + 2}$ and returning the bottom-right entry of the result.
                This works because $A$ encodes the base cases of the function
                and computes an update in multiplication

                \[
                    \left[\begin{array}{cc}
                        \textsc{Steps}(n + 1) & \textsc{Steps}(n) \\
                        \textsc{Steps}(n)     & \textsc{Steps}(n - 1)
                    \end{array}\right] \times
                    \left[\begin{array}{cc}
                        1 & 1 \\
                        1 & 0
                    \end{array}\right] =
                    \left[\begin{array}{cc}
                        \textsc{Steps}(n + 2) & \textsc{Steps}(n + 1) \\
                        \textsc{Steps}(n + 1) & \textsc{Steps}(n)
                    \end{array}\right]
                \]

                and on an arbitrary input of $n \geq 1$, computing $A^{n + 2}$
                will result in a $2 \times 2$ matrix with $\textsc{Steps}(n)$
                in its bottom-right corner, which is returned. 
                
        \end{itemize}
    \end{homeworkAnswer}
\end{homeworkProblem}